\documentclass{beamer}
\usetheme[color=screen]{UniBern}

%\setbeameroption{show notes}
%\includeonlyframes{current}

\usepackage{lmodern}
\usepackage[english]{babel}
\usepackage{microtype}
\usepackage{textcomp}
\usepackage[backend=biber, style=numeric, url=false, isbn=false, maxbibnames=1, sorting=none]{biblatex}
	\addbibresource{../../Documents/library.bib}
\usepackage{graphicx}
\usepackage{tikz}
	\usetikzlibrary{arrows.meta,shapes}
\usepackage[detect-all=true, range-phrase=--, range-units=single]{siunitx}
\usepackage{csquotes}
\usepackage{animate}
\usepackage[absolute,overlay]{textpos} %for the \source{} command
\usepackage{gitinfo2}
\usepackage{xspace}
\usepackage[normalem]{ulem}
\usepackage{hyperref}

% Setup some often used commands
\newcommand{\imsize}{\linewidth}
\newlength\imagewidth % needed for scalebars
\newlength\imagescale % ditto
\newcommand{\uct}{\si{\micro}CT\xspace}
\newcommand{\uaf}{\si{\micro}AngioFil\xspace}

% Acknowledge random stuff
\newcommand{\source}[1]{%http://tex.stackexchange.com/a/48485/828
	\begin{textblock*}{4cm}(8.7cm,8.6cm)%
		\begin{beamercolorbox}[ht=0.5cm,right]{framesource}%
			\tiny\usebeamerfont{framesource}\usebeamercolor[fg]{framesource} Source: {#1}%
		\end{beamercolorbox}%
	\end{textblock*}%
}

% define 'unibe' color
\definecolor{unibe}{RGB}{156,189,222}

% Show current section at begin of sections
\AtBeginSection[]{
	\begin{frame}{Outline}
	\small \tableofcontents[currentsection, hideothersubsections]
	\end{frame} 
}

% Define us a nice footer with the 'necessary' info
\defbeamertemplate{footline}{unibe}
{%
\usebeamercolor[fg]{page number in head/foot}%
	\usebeamerfont{page number in head/foot}%
	\hspace*{\fill}%
	\insertshortauthor%
	\hspace*{\fill}%|\hspace*{\fill}%
	\insertshorttitle%
	\hspace*{\fill}%|\hspace*{\fill}%
	v.~\gitAbbrevHash%
	\hspace*{\fill}%|\hspace*{\fill}%
	\insertframenumber\,/\,\insertpresentationendpage%
	\hspace*{\fill}%
	\vskip2pt%
}
\setbeamertemplate{footline}[unibe]

% Format bibliography for beamer
% http://tex.stackexchange.com/a/10686/828
\renewbibmacro{in:}{}
% % http://tex.stackexchange.com/a/13076/828
% \AtEveryBibitem{\clearfield{title}}
\AtEveryBibitem{\clearfield{journaltitle}}
\AtEveryBibitem{\clearfield{pages}}
\AtEveryBibitem{\clearfield{volume}}
\AtEveryBibitem{\clearfield{number}}
\AtEveryBibitem{\clearfield{editors}}

% Subtitle and other informations
\subtitle{Several examples as well as details in lung and brain imaging}
\author[David Haberthür]{David Haberthür \and\tiny Adolfo Odriozola \and Ruslan Hlushchuk \and Valentin Djonov}
\institute{Institute of Anatomy, University of Bern}
\date{February 9, 2017\\Internal Seminar}

\begin{document}
\title[\si{\micro}CT in biological studies]{\si{\micro}CT-imaging at the Institute of Anatomy} % http://tex.stackexchange.com/a/144445/828

{%
% No footline on title page: http://tex.stackexchange.com/a/18829/828
	\setbeamertemplate{footline}{}%
	\begin{frame}%
		\titlepage%
	\end{frame}%
}
\setcounter{framenumber}{0}

\begin{frame}{Contents}
	\tableofcontents
\end{frame}

\renewcommand{\imsize}{0.95\textheight}
\begin{frame}{Computed tomography}
	\begin{columns}
		\begin{column}{0.7\linewidth}
			\includegraphics[width=\imsize]{img/CT_PRINCI_PB}
				\source{\href{https://commons.wikimedia.org/wiki/File:CT_PRINCI_PB.jpg}{enwp.org/tomography}}	
		\end{column}
		\begin{column}{0.4\linewidth}
			\begin{enumerate}
				\item Object
				\item Parallel light source
				\item Screen
				\item Transmitted beam
				\item Datum circle
				\item Origin
				\item 1D image
			\end{enumerate}
		\end{column}
	\end{columns}
\end{frame}

\section{Current and ongoing projects}
\begin{frame}{Zebrafish}
	\renewcommand{\imsize}{\paperwidth}%
	\pgfmathsetlength{\imagewidth}{\imsize}%
	\pgfmathsetlength{\imagescale}{\imagewidth/1651}%
	\def\x{1020}% scalebar-x starting at golden ratio of image width of 1651px = 1020
	\def\y{391}% scalebar-y at 90% of image height of 434px = 391
	\makebox[\linewidth]{%
	\begin{tikzpicture}[x=\imagescale,y=-\imagescale]%
		\node[anchor=north west, inner sep=0pt, outer sep=0pt] at (0,0) {\includegraphics[width=\imagewidth]{img/{{zebrafish_rec_voi_side}}}};
		% 1615px = 35.48 > 100px = 2198um > 23px = 500um, 5px = 100um
		%\draw[|-|,blue,thick] (20,154) -- (1634,127) node [sloped,midway,above,fill=white,semitransparent,text opacity=1] {\SI{35.487480000000005}{\milli\meter} (1615px) TEMPORARY!};
		\draw[|-|] (\x,\y) -- (\x+234.19,\y) node [midway,above] {\SI{5}{\milli\meter}};
	\end{tikzpicture}%
	}
	Visualization of a tomographic scan of a zebrafish, fixed in \SI{4}{\percent} PFA.
\end{frame}

\begin{frame}
	\frametitle{Rat head}
	\renewcommand{\imsize}{\paperwidth}%
	\pgfmathsetlength{\imagewidth}{\imsize}%
	\pgfmathsetlength{\imagescale}{\imagewidth/1400}%
	\def\x{865}% scalebar-x starting at golden ratio of image width of 1400px = 865
	\def\y{560}% scalebar-y at 90% of image height of 622px = 560
	\makebox[\linewidth]{%
	\begin{tikzpicture}[x=\imagescale,y=-\imagescale]%
		\node[anchor=north west, inner sep=0pt, outer sep=0pt] at (0,0) {\includegraphics[width=\imagewidth]{./img/{{ratwholehead_rec_side}}}};
		%\spy [red] on (1100,322) in node at (0,0) [anchor=north west];
		% 1358px = 47.76mm > 100px = 3517um > 14px = 500um, 3px = 100um
		%\draw[|-|,blue,thick] (23,291) -- (1380,311) node [sloped,midway,above,fill=white,semitransparent,text opacity=1] {\SI{47.76}{\milli\meter} (1358px) TEMPORARY!};
		\draw[|-|] (\x,\y) -- (\x+140,\y) node [midway,above] {\SI{5}{\milli\meter}};
	\end{tikzpicture}%
	}
	Visualization of a tomographic scan of a rat head, instilled with \uaf and fixed in \SI{4}{\percent} PFA.
\end{frame}

\begin{frame}{Brain vasculature}
	\renewcommand{\imsize}{\linewidth}%
	\begin{columns}
		\begin{column}{0.618\linewidth}
			\pgfmathsetlength{\imagewidth}{\imsize}%
			\pgfmathsetlength{\imagescale}{\imagewidth/1950}%
			\def\x{1205}% scalebar-x starting at golden ratio of image width of 1950px = 1205
			\def\y{1230}% scalebar-y at 90% of image height of 1367px = 1230
			\begin{tikzpicture}[x=\imagescale,y=-\imagescale]
				\node[anchor=north west, inner sep=0pt, outer sep=0pt] at (0,0) {\includegraphics[width=\imagewidth]{img/Ebene_25}};
				% 1949px = 20.0mm > 100px = 1026um > 49px = 500um, 10px = 100um
				%\draw[|-|,blue,thick] (39,1222) -- (1722,239) node [sloped,midway,above,fill=white,semitransparent,text opacity=1] {\SI{20.0}{\milli\meter} (1949px) TEMPORARY!};
				\draw[|-|] (\x,\y) -- (\x+490,\y) node [midway,above] {\SI{5}{\milli\meter}};
				\node [anchor=center] (cb) at (2000,0) {Cerebellum};
				\draw [thick,double,draw=unibe,double=black] (cb.west) to [out=-180, in=0] (1482,144);
				\node [anchor=center] (bs) at (2250,1500) {Brain stem};
				\draw [thick,double,draw=unibe,double=black] (bs.west) to [out=180, in=-45] (1705,725);
				\node [anchor=center, text width=2cm, align=center] (ob) at (550,1500) {Olfactory bulbs};
				\draw [thick,double,draw=unibe,double=black] (ob.west) to [out=180, in=180] (210,880);
				\draw [thick,double,draw=unibe,double=black] (ob.west) to [out=180, in=180] (298,1116);
			\end{tikzpicture}
		\end{column}		
		\begin{column}{0.382\linewidth}
			Visualization of a tomographic scan of a \uaf-filled mouse brain.
		\end{column}
	\end{columns}
\end{frame}

\begin{frame}
	\frametitle{Help for the Sekretariat}
	\renewcommand{\imsize}{0.618\linewidth}%
		\animategraphics[autoplay,palindrome,width=\imsize]{24}{img/movie_spider_7um/spider_07um_rec00000}{001}{501}
\end{frame}

\section{Assessing tumor/metastasis load in lungs}
\begin{frame}
	\frametitle{Tumor metastasis}
	\begin{itemize}
		\item \emph{Problem:} Tumor load?
		\pause
		\item \emph{Solution:} Stereology\pause\xspace on \uct data!
		\item[]
		\pause
		\item KP-TNIK mice \cite{DuPage2009} from collaboration with DKF
		\item Stereology as part of \href{https://www.anaintra.unibe.ch/studenten_view.php?filter=aktuell}{Masters Thesis} of \href{http://www.anaweb.ch/ueber_uns/team/detail/index_ger.php?id=485}{Andreas Gübeli}
	\end{itemize}
	\note{%
		Unsere Maus nennt sich KP-Mouse model.
		Wir haben ein conditional knockout (Null-Allel) für p53 und eine conditional overactivation of KRAS (G12D point mutation).
		Conditional heisst, dass wir im "Normalzustand" physiologische Expressionen von KRAS und p53 haben.
		Bei Darmtumore schaut man darauf, ob KRAS mutiert ist.
		KRAS-Mutationen führen zu unkontrollierter Zellteilung.
		p53 ist Tumorsuppressor.
		Diese 2 Mutationen führen dann in der Maus innerhalb von wenigen Wochen/Monaten zu Tumoren.
		Wenn wir also am Schluss eine KP-TNIK Maus haben und diese mit einem Cre Exprimierenden Virus in der Lunge behandeln, wird simultan TNIK und p53 deaktiviert und KRAS überaktiviert.
		Haus-Pharmazeutin sagt, dass KO mehr Tumore haben muss.}
\end{frame}

\begin{frame}
	\frametitle{Tumor load in lungs, KP-TNIK mice}
	\renewcommand{\imsize}{0.33\linewidth}
	\begin{columns}
		\begin{column}{0.8\linewidth}
			\pgfmathsetlength{\imagewidth}{\imsize}%
			\pgfmathsetlength{\imagescale}{\imagewidth/800}%
			\def\x{400-125}% scalebar-x starting at golden ratio of image width of 800px = 494
			\def\y{720}% scalebar-y at 90% of image height of 800px = 720
			\begin{tikzpicture}[x=\imagescale,y=-\imagescale]
			\node[anchor=north west, inner sep=0pt, outer sep=0pt] at (0,0) {\includegraphics[width=\imagewidth]{./img/ochsenbein/{{MAX_--11}}}};
			% 800px = 16.0mm > 100px = 2000um > 25px = 500um, 5px = 100um
			%\draw[|-|,blue,thick] (0,400) -- (800,400) node [sloped,midway,above,fill=white,semitransparent,text opacity=1] {\SI{16.0}{\milli\meter} (800px) TEMPORARY!};
			%\draw[|-|] (\x,\y) -- (\x+250,\y) node [midway,above] {\SI{5}{\milli\meter}};
			\end{tikzpicture}%
			\begin{tikzpicture}[x=\imagescale,y=-\imagescale]
			\node[anchor=north west, inner sep=0pt, outer sep=0pt] at (0,0) {\includegraphics[width=\imagewidth]{./img/ochsenbein/{{MAX_--12}}}};
			%\draw[|-|] (\x,\y) -- (\x+250,\y) node [midway,above] {\SI{5}{\milli\meter}};
			\end{tikzpicture}%
			\begin{tikzpicture}[x=\imagescale,y=-\imagescale]
			\node[anchor=north west, inner sep=0pt, outer sep=0pt] at (0,0) {\includegraphics[width=\imagewidth]{./img/ochsenbein/{{MAX_--13}}}};
			\draw[|-|] (\x,\y) -- (\x+250,\y) node [midway,above] {\SI{5}{\milli\meter}};
			\end{tikzpicture}%
			\vspace{0.25cm}%
			\\%
			\begin{tikzpicture}[x=\imagescale,y=-\imagescale]
			\node[anchor=north west, inner sep=0pt, outer sep=0pt] at (0,0) {\includegraphics[width=\imagewidth]{./img/ochsenbein/{{MAX_wt11}}}};
			%\draw[|-|] (\x,\y) -- (\x+250,\y) node [midway,above] {\SI{5}{\milli\meter}};
			\end{tikzpicture}%
			\begin{tikzpicture}[x=\imagescale,y=-\imagescale]
			\node[anchor=north west, inner sep=0pt, outer sep=0pt] at (0,0) {\includegraphics[width=\imagewidth]{./img/ochsenbein/{{MAX_wt12}}}};
			%\draw[|-|] (\x,\y) -- (\x+250,\y) node [midway,above] {\SI{5}{\milli\meter}};
			\end{tikzpicture}%
			\begin{tikzpicture}[x=\imagescale,y=-\imagescale]
			\node[anchor=north west, inner sep=0pt, outer sep=0pt] at (0,0) {\includegraphics[width=\imagewidth]{./img/ochsenbein/{{MAX_wt13}}}};
			\draw[|-|] (\x,\y) -- (\x+250,\y) node [midway,above] {\SI{5}{\milli\meter}};
			\end{tikzpicture}%
		\end{column}
		\begin{column}{0.15\textwidth}
			\begin{itemize}
				\item KO
				\item WT
			\end{itemize}
		\end{column}
	\end{columns}
\end{frame}

\begin{frame}
	\frametitle{Tumor load in lungs, KP-TNIK mice.}
	\renewcommand{\imsize}{\linewidth}%
	\begin{columns}
		\begin{column}{0.8\linewidth}
		\animategraphics[autoplay,palindrome,width=\imsize]{12}{../../Documents/Collaborations/DKF_Lung/movieframes/out-}{001}{255}
		\end{column}
		\begin{column}{0.15\textwidth}
			\begin{itemize}
				\item KO
				\item WT
			\end{itemize}
		\end{column}
	\end{columns}
\end{frame}

\section{Lung fibrosis}
\begin{frame}
	\frametitle{Lung fibrosis}
	\begin{itemize}
		\item \emph{Problem:} Lung fibrosis grading \cite{Ashcroft1988a} and correct sampling for proper assessment?
		\pause
		\item \emph{Solution:} Correlative imaging with \uct, EM (and Histology)!
		\item[]
		\pause
	\item \uct is a tool to help grade fibrosis \cite{DeLanghe2012,Rodt2010}
		\begin{itemize}
			\item Detect and mark fibrosis regions in 3D
			\item Get indications where to perform the sampling
			\item Still limited resolution
		\end{itemize}
	\end{itemize}
\end{frame}

\begin{frame}
	\frametitle{Multimodal imaging: \uct and 3View, correlated imaging}
	\renewcommand{\imsize}{0.25\linewidth}
	\begin{itemize}
		\item Fibrosis grading should be done on ultra fine level \cite{Hubner2008}
		\pause
		\item \uct combined with serial block-face scanning electron microscopy (\href{http://www.gatan.com/products/sem-imaging-spectroscopy/3view-system}{3View})
		\item Registration of datasets; from overview to ultrafine structure
		\item Combination with histology
		\pause
		\item Multimodal has been done before, with luck and long, hard work, not in full 3D \cite{Haberthuer2009}
	\end{itemize}
\end{frame}

\begin{frame}
	\frametitle{Overview scan with \SI{20}{\micro\meter} pixel size}
	\renewcommand{\imsize}{0.53\linewidth}
	\pgfmathsetlength{\imagewidth}{\imsize}%
	\pgfmathsetlength{\imagescale}{\imagewidth/497}%
	\def\x{614/2}% scalebar-x starting at golden ratio of image width of 993px = 614
	\def\y{859/2}% scalebar-y at 90% of image height of 954px = 859
	\begin{tikzpicture}[x=\imagescale,y=-\imagescale]
		\node[anchor=north west, inner sep=0pt, outer sep=0pt] at (0,0) {\animategraphics[autoplay, palindrome, width=\imsize]{24}{./img/movie_tumor_20um/mouse_tumor_rec0}{001}{461}};
		% 698px = 16.12mm > 100px = 2309um > 22px = 500um, 4px = 100um
		%\draw[|-|,blue,thick] (89,571) -- (787,576) node [sloped,midway,above,fill=white,semitransparent,text opacity=1] {\SI{16.12}{\milli\meter} (806px) TEMPORARY!};
		\draw[|-|] (\x,\y) -- (\x+217/2,\y) node [midway,above] {\SI{5}{\milli\meter}};
	\end{tikzpicture}%
\end{frame}

\begin{frame}
	\frametitle{Overview, Detail and 3View}
	\renewcommand{\imsize}{0.5\linewidth}
	\pgfmathsetlength{\imagewidth}{\imsize}%
	\pgfmathsetlength{\imagescale}{\imagewidth/7776}%
	\def\x{4806}% scalebar-x starting at golden ratio of image width of 7776px = 4806
	\def\y{6998}% scalebar-y at 90% of image height of 7776px = 6998
	\begin{tikzpicture}[x=\imagescale,y=-\imagescale,remember picture]
		\node[anchor=north west, inner sep=0pt, outer sep=0pt] at (0,0) {\includegraphics[width=\imagewidth]{./img/fibrosis/Mouse_Tumor_IR_rec00001610}};
		% 7776px = 23.328mm > 100px = 300um > 167px = 500um, 33px = 100um
		%\draw[|-|,blue,thick] (0,3888) -- (7776,3888) node [sloped,midway,above,fill=white,semitransparent,text opacity=1] {\SI{23.328}{\milli\meter} (7776px) TEMPORARY!};
		\draw[|-|,white] (\x,\y) -- (\x+1670,\y) node [midway,above] {\SI{5}{\milli\meter}};
		\node<3>[red,draw,circle,inner sep=6pt] (overview) at (1500,4160) {};
	\end{tikzpicture}%
	\pause
	\pgfmathsetlength{\imagewidth}{\imsize}%
	\pgfmathsetlength{\imagescale}{\imagewidth/3812}%
	\def\x{2356}% scalebar-x starting at golden ratio of image width of 3812px = 2356
	\def\y{3431}% scalebar-y at 90% of image height of 3812px = 3431
	\begin{tikzpicture}[x=\imagescale,y=-\imagescale, remember picture]
		\node<2->[anchor=north west, inner sep=0pt, outer sep=0pt] at (0,0) {\includegraphics[width=\imagewidth]{./img/fibrosis/uct_Cube}};
		%\spy [red] on (3512,3512) in node at (0,0) [anchor=north west];
		% 3812px = 1.906mm > 100px = 50um > 1000px = 500um, 200px = 100um
		%\draw[|-|,blue,thick] (0,1906) -- (3812,1906) node [sloped,midway,above,fill=white,semitransparent,text opacity=1] {\SI{1.906}{\milli\meter} (3812px) TEMPORARY!};
		\draw[|-|,white] (\x,\y) -- (\x+1000,\y) node [midway,above] {\SI{500}{\micro\meter}};
		\node<3> (detail-1) at (2124,336) {};
		\node<3> (detail-2) at (1470,3222) {};
	\end{tikzpicture}%
	\begin{tikzpicture}[remember picture,overlay]
		\draw<3>[dashed,red] (overview) -- (detail-1);
		\draw<3>[dashed,red] (overview) -- (detail-2);
	\end{tikzpicture}
\end{frame}

\begin{frame}
	\frametitle{Overview, Detail and 3View}
	\renewcommand{\imsize}{0.5\linewidth}
	\pgfmathsetlength{\imagewidth}{\imsize}%
	\pgfmathsetlength{\imagescale}{\imagewidth/8192}%
	\def\x{5063}% scalebar-x starting at golden ratio of image width of 8192px = 5063
	\def\y{6313}% scalebar-y at 90% of image height of 7014px = 6313
	\begin{tikzpicture}[x=\imagescale,y=-\imagescale, remember picture]
		\node[anchor=north west, inner sep=0pt, outer sep=0pt] at (0,0) {\includegraphics[width=\imagewidth]{./img/fibrosis/3View_Overview}};
	% 8192px = 0.9486336mm > 100px = 12um > 4318px = 500um, 864px = 100um
	%\draw[|-|,blue,thick] (0,3507) -- (8192,3507) node [sloped,midway,above,fill=white,semitransparent,text opacity=1] {\SI{0.9486336}{\milli\meter} (8192px) TEMPORARY!};
	\draw[|-|,white] (\x,\y) -- (\x+864,\y) node [midway,above] {\SI{100}{\micro\meter}};		\node<3>[red,draw,circle,inner sep=8pt] (overview) at (792,3280) {};
\end{tikzpicture}%
	\pause
\pgfmathsetlength{\imagewidth}{\imsize}%
\pgfmathsetlength{\imagescale}{\imagewidth/8192}%
\def\x{5063}% scalebar-x starting at golden ratio of image width of 8192px = 5063
\def\y{7373}% scalebar-y at 90% of image height of 8192px = 7373
\begin{tikzpicture}[x=\imagescale,y=-\imagescale, remember picture]
		\node[anchor=north west, inner sep=0pt, outer sep=0pt] at (0,0) {\includegraphics[width=\imagewidth]{./img/fibrosis/3View_Detail}};;
	% 8192px = 0.098304mm > 100px = 1um > 41667px = 500um, 8333px = 100um
	%\draw[|-|,blue,thick] (0,4096) -- (8192,4096) node [sloped,midway,above,fill=white,semitransparent,text opacity=1] {\SI{0.098304}{\milli\meter} (8192px) TEMPORARY!};
		\draw[|-|,white] (\x,\y) -- (\x+833.3,\y) node [midway,above] {\SI{10}{\micro\meter}};
		\node<3> (detail-2) at (4096,250) {};
		\node<3> (detail-2) at (4096,8192-250) {};
	\end{tikzpicture}%
	\begin{tikzpicture}[remember picture,overlay]
		\draw<3>[dashed,red] (overview) -- (detail-1);
		\draw<3>[dashed,red] (overview) -- (detail-2);
	\end{tikzpicture}
\end{frame}

\begin{frame}
	\frametitle{Overview, Detail and 3View}
	\renewcommand{\imsize}{0.6\linewidth}
	\pgfmathsetlength{\imagewidth}{\imsize}%
	\pgfmathsetlength{\imagescale}{\imagewidth/8192}%
	\def\x{5063}% scalebar-x starting at golden ratio of image width of 8192px = 5063
	\def\y{6313}% scalebar-y at 90% of image height of 7014px = 6313
	\begin{tikzpicture}[x=\imagescale,y=-\imagescale]
		\node[anchor=north west, inner sep=0pt, outer sep=0pt] at (0,0) {\includegraphics[width=\imagewidth]{./img/fibrosis/3View_Registration}};
		% 8192px = 0.9486336mm > 100px = 12um > 4318px = 500um, 864px = 100um
		%\draw[|-|,blue,thick] (0,3507) -- (8192,3507) node [sloped,midway,above,fill=white,semitransparent,text opacity=1] {\SI{0.9486336}{\milli\meter} (8192px) TEMPORARY!};
		\draw[|-|,white] (\x,\y) -- (\x+864,\y) node [midway,above] {\SI{100}{\micro\meter}};
	\end{tikzpicture}%
\end{frame}

\begin{frame}
	\frametitle{Overview, Detail and 3View}
	\renewcommand{\imsize}{0.5\linewidth}
	\pgfmathsetlength{\imagewidth}{\imsize}%
	\pgfmathsetlength{\imagescale}{\imagewidth/3812}%
	\def\x{2356}% scalebar-x starting at golden ratio of image width of 3812px = 2356
	\def\y{3431}% scalebar-y at 90% of image height of 3812px = 3431
	\begin{tikzpicture}[x=\imagescale,y=-\imagescale]
		\node[anchor=north west, inner sep=0pt, outer sep=0pt] at (0,0) {\includegraphics[width=\imagewidth]{./img/fibrosis/uCT_Cube_with_3View_Registration_Average}};
		% 3812px = 11.436mm > 100px = 300um > 167px = 500um, 33px = 100um
		%\draw[|-|,blue,thick] (0,1906) -- (3812,1906) node [sloped,midway,above,fill=white,semitransparent,text opacity=1] {\SI{11.436}{\milli\meter} (3812px) TEMPORARY!};
		\draw[|-|,white] (\x,\y) -- (\x+1000,\y) node [midway,above] {\SI{500}{\micro\meter}};
	\end{tikzpicture}%
\end{frame}

\section{Quantitative assessment of brain tumor radiation\newline treatment}
\label{sec:grenoble}
\begin{frame}
	\frametitle{Brain tumor radiation treatment}
	\begin{itemize}
		\item \emph{Problem:} Tumor angiogenesis/radiation therapy?
		\pause
		\item \emph{Solution:} Nondestructive imaging with \uct and quantitative assessment of tomographic data!
		\item[]
		\pause
		\item Judah Folkman: \blockquote[\cite{Sherwood1971}]{Tumors that exist in the dormant state have not become vascularised.}
		\item Anti-Angiogenesis as tumor therapy
		\begin{itemize}
			\item \href{https://clinicaltrials.gov/ct2/results?term=antiangiogenic}{\textgreater4000 clinical trials}
			\item Results \sout{disappointing} unsatisfactory
			\item New, powerful and simple treatment strategies are needed
		\end{itemize}
		\item Microbeam radiation therapy
		\begin{itemize}
			\item Delivery of very high dose (\SIrange{100}{5000}{\gray}) in less than \SI{1}{\second}.
			\item \emph{Excellent} survival rate \cite{Laissue1998}
		\end{itemize}
	\end{itemize}
\end{frame}

\begin{frame}
	\frametitle{Rat brain tumors}
	\begin{columns}
		\begin{column}{0.5\linewidth}
			\begin{itemize}
				\item Induced tumors
				\item 9L Gliosarcoma \cite{Bouchet2014}
				\item Microbeam and broadbeam treatment vs.\ control at three time points
				\item \uaf-filled brain vasculature
				\item Nondestructive extraction of
				\begin{itemize}
					\item Vascular density
					\item Vessel surface
					\item Intra-tumoral microvessel density (IMD, \cite{Hasan2002})
				\end{itemize}
			\end{itemize}
		\end{column}
		\renewcommand{\imsize}{1\linewidth}
		\begin{column}{0.29\linewidth}
			\pgfmathsetlength{\imagewidth}{\imsize}%
			\pgfmathsetlength{\imagescale}{\imagewidth/299}%
			\def\x{185}% scalebar-x starting at golden ratio of image width of 299px = 185
			\def\y{460}% scalebar-y at 90% of image height of 546px = 491
			\begin{tikzpicture}[x=\imagescale,y=-\imagescale]
				\node[anchor=north west, inner sep=0pt, outer sep=0pt] at (0,0) {\includegraphics[width=\imagewidth]{./img/{{macro_brain_d24_ctrl}}}};
				% 500px = 30.0mm > 100px = 6004um > 8px = 500um, 2px = 100um
				%\draw[|-|,blue,thick] (179,523) -- (121,27) node [sloped,midway,above,fill=white,semitransparent,text opacity=1] {\SI{30.0}{\milli\meter} (500px) TEMPORARY!};
				\draw[|-|] (\x,\y) -- (\x+83,\y) node [midway,above] {\SI{5}{\milli\meter}};
				\pause
				\draw<2>[red, thick] (203,192) circle (50);
			\end{tikzpicture}%
		\end{column}%
		\begin{column}{0.29\linewidth}%
			\pgfmathsetlength{\imagewidth}{\imsize}%
			\pgfmathsetlength{\imagescale}{\imagewidth/299}%
			\def\x{185}% scalebar-x starting at golden ratio of image width of 299px = 185
			\def\y{460}% scalebar-y at 90% of image height of 511px = 460
			\begin{tikzpicture}[x=\imagescale,y=-\imagescale]
				\node[anchor=north west, inner sep=0pt, outer sep=0pt] at (0,0) {\includegraphics[width=\imagewidth]{./img/{{macro_brain_d24_mrt}}}};
				% 463px = 30.0mm > 100px = 6478um > 8px = 500um, 2px = 100um
				%\draw[|-|,blue,thick] (163,479) -- (162,16) node [sloped,midway,above,fill=white,semitransparent,text opacity=1] {\SI{30.0}{\milli\meter} (463px) TEMPORARY!};
				\draw[|-|] (\x,\y) -- (\x+77,\y) node [midway,above] {\SI{5}{\milli\meter}};
				\draw<2>[red, thick] (199,187) circle (30);
			\end{tikzpicture}%
		\end{column}		
	\end{columns}	
\end{frame}

\begin{frame}
	\frametitle{Rat brain tumors}
	\begin{itemize}
		\item Scan (\SI{18}{\hour})
		\item Reconstruct (\SIrange{2}{4}{\hour})
		\item Delineate ROI (\SI{1}{\hour})
		\item Threshold
		\item Calculate Tumor and vessel volume
	\end{itemize}
\end{frame}

\begin{frame}
	\frametitle{Rat brain tumors}
	\renewcommand{\imsize}{\linewidth}
    \begin{columns}
		\begin{column}{0.33\paperwidth}
            \pgfmathsetlength{\imagewidth}{\imsize}%
            \pgfmathsetlength{\imagescale}{\imagewidth/535}%
            \def\x{1001/3}% scalebar-x starting at golden ratio of image width of 1080px
            \def\y{978/3}% scalebar-y at 90% of image height of 1620px
            \begin{tikzpicture}[x=\imagescale,y=-\imagescale]
                \node[anchor=north west, inner sep=0pt, outer sep=0pt] at (0,0) {\animategraphics[autoplay,palindrome,width=\imsize]{24}{./img/movie_b09/b09_ctrl_d24t14_ir_rec0000_voi_0}{001}{501}};
                % 618px = 6.18mm > 100px = 1000um > 50px = 500um, 10px = 100um
                %\draw[|-|,blue,thick] (0,346) -- (618,346) node [sloped,midway,above,fill=white,semitransparent,text opacity=1] {\SI{6.18}{\milli\meter} (618px) TEMPORARY!};
                \draw[|-|] (\x,\y) -- (\x+100/3,\y) node [right] {\SI{1}{\milli\meter}};
            \end{tikzpicture}%
            \centering

            Control
		\end{column}
		\begin{column}{0.33\paperwidth}
            \pgfmathsetlength{\imagewidth}{\imsize}%
            \pgfmathsetlength{\imagescale}{\imagewidth/535}%
            \def\x{1001/3}% scalebar-x starting at golden ratio of image width of 1080px
            \def\y{978/3}% scalebar-y at 90% of image height of 1620px		
            \begin{tikzpicture}[x=\imagescale,y=-\imagescale]
                \node[anchor=north west, inner sep=0pt, outer sep=0pt] at (0,0) {\animategraphics[autoplay,palindrome,width=\imsize]{24}{./img/movie_b18/b18_mrt_d24t14_rec0000_voi_0}{001}{501}};
                %\spy [red] on (462,234) in node at (0,0) [anchor=north west];
                % 762px = 7.62mm > 100px = 1000um > 50px = 500um, 10px = 100um
                %\draw[|-|,blue,thick] (0,267) -- (762,267) node [sloped,midway,above,fill=white,semitransparent,text opacity=1] {\SI{7.62}{\milli\meter} (762px) TEMPORARY!};
                \draw[|-|] (\x,\y) -- (\x+100/3,\y) node [right] {\SI{1}{\milli\meter}};
            \end{tikzpicture}%
            \centering

            Microbeam
		\end{column}
		\begin{column}{0.33\paperwidth}
            \pgfmathsetlength{\imagewidth}{\imsize}%
            \pgfmathsetlength{\imagescale}{\imagewidth/535}%
            \def\x{1001/3}% scalebar-x starting at golden ratio of image width of 1080px
            \def\y{978/3}% scalebar-y at 90% of image height of 1620px		
            \begin{tikzpicture}[x=\imagescale,y=-\imagescale]
                \node[anchor=north west, inner sep=0pt, outer sep=0pt] at (0,0) {\animategraphics[autoplay,palindrome,width=\imsize]{24}{./img/movie_b68/b68_bb_d24t14_rec0000_voi_0}{001}{501}};
                % 992px = 9.92mm > 100px = 1000um > 50px = 500um, 10px = 100um
                %\draw[|-|,blue,thick] (0,320) -- (992,320) node [sloped,midway,above,fill=white,semitransparent,text opacity=1] {\SI{9.92}{\milli\meter} (992px) TEMPORARY!};
                \draw[|-|] (\x,\y) -- (\x+100/3,\y) node [right] {\SI{1}{\milli\meter}};
            \end{tikzpicture}%
            \centering

            Broadbeam
		\end{column}
	\end{columns}
\end{frame}

\renewcommand{\imsize}{0.8\textheight}
\begin{frame}
	\frametitle{Tumor volume}
	\includegraphics[height=\imsize]{../../Documents/Brain-Grenoble/fig/tumor_volume}
\end{frame}

\begin{frame}
	\frametitle{Tumor volume}
	\includegraphics[height=\imsize]{../../Documents/Brain-Grenoble/fig/tumor_volume_day}
\end{frame}

\begin{frame}
	\frametitle{Vessel ratio}
	\includegraphics[height=\imsize]{../../Documents/Brain-Grenoble/fig/vessel_vol_per_tumor_volume}
\end{frame}

\begin{frame}
	\frametitle{Vessel ratio}
	\includegraphics[height=\imsize]{../../Documents/Brain-Grenoble/fig/vessel_vol_per_tumor_volume_day}
\end{frame}

\begin{frame}
	\frametitle{Summary}
	\begin{itemize}
		\item \uct is a powerful method to get both pretty and meaningful insight into a broad kind of samples
		\item Non-destructiveness makes it a complementary method to established assessment methods (Stereology, SEM, Histology)
		\item Three-dimensional, numerical data enables quantitative assessment of sample properties
		\item[]
		\pause
		\item Come and ask us if you want to look into your samples!
	\end{itemize}
\end{frame}

\begin{frame}
	\frametitle{Thanks}
	\begin{itemize}
		\item Topographic and clinical Anatomy
		\item Audrey Bouchet
		\item Jean Laissue
		\item ESRF Grenoble
		\item SNF
		\pause
		\item You, for listening
		\item[]
		\pause
		\item Questions?
	\end{itemize}
\end{frame}

\begin{frame}[allowframebreaks]
	\frametitle{References}
	\renewcommand*{\bibfont}{\scriptsize}
	\setbeamertemplate{bibliography item}{\insertbiblabel}
	\printbibliography
\end{frame}

\end{document}
